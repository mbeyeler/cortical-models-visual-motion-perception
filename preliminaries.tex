\thesistitle{Cortical Neural Network Models of Visual Motion Perception for Decision-Making and Reactive Navigation}

\degreename{Doctor of Philosophy}

% Use the wording given in the official list of degrees awarded by UCI:
% http://www.rgs.uci.edu/grad/academic/degrees_offered.htm
\degreefield{Computer Science}

% Your name as it appears on official UCI records.
\authorname{Michael Beyeler}

% Use the full name of each committee member.
\committeechairone{Professor Jeffrey L. Krichmar}
\committeechairtwo{Professor Nikil D. Dutt}
\othercommitteemembers
{
  Professor Charless C. Fowlkes
}

\degreeyear{2016}

\copyrightdeclaration
{
  {\copyright} {\Degreeyear} \Authorname
}

% If you have previously published parts of your manuscript, you must list the
% copyright holders; see Section 3.2 of the UCI Thesis and Dissertation Manual.
% Otherwise, this section may be omitted.
\prepublishedcopyrightdeclaration
{
	
	Chapter \ref{ch:SNN} {\copyright} 2014, 2015 IEEE \\
    Portion of Chapter \ref{ch:ME} {\copyright} 2014 Springer \\
    Portion of Chapter \ref{ch:MT} {\copyright} 2014 Springer \\
    Chapter \ref{ch:ABR} {\copyright} 2015 IEEE \\
	All other materials {\copyright} {\Degreeyear} \Authorname
}

% The dedication page is optional.
% \dedications
% {
%   (Optional dedication page)
  
%   To ...
% }

\acknowledgments
{
	First and foremost I would like to thank my PhD advisors, Profs.
    Jeff Krichmar and Nik Dutt.
    It was Jeff who gave me the chance to conduct my Master's Thesis in his lab
    back when I first dared to cross the pond and move to Irvine,
    and it was Nik who subsequently stepped
    in and helped us revive the ``Computational Neuroscience'' track in the
    Computer Science department at UCI.
    Both of them were excellent mentors demonstrating perpetual energy, 
    expertise, and vision in research.
    I am thankful for everything they have taught me,
    for their patience and kindness,
    and for the freedom they gave me to explore new ideas.
    I also thank Prof. Charless Fowlkes for his efforts in pre-reviewing my
    thesis work and challenging me to approach my ideas from different angles
    and explain my work in different ways.
    
    A big thanks must go to my labmates and co-authors of the original publications.
    Nicolas Oros: thank you for teaching me how to break big projects into
    feasible steps, and for developing the Android Based Robotics (ABR) platform.
    Ting-Shuo Chou: thank you for your \emph{joie de vivre}, 
    your endless programming knowledge, 
    and for teaching me how to say ``piece of cake'' in Mandarin.
    Kristofor Carlson: thank you for being there and saying things no one expects,
    for the 2AM writing sessions, for Science Fridays, 
    and of course your (un)disputable soccer expertise.
    Micah Richert, whose legacy most of my work is based on: thank you for your
    efforts developing an early version of CARLsim,
    for steering me towards studying visual motion perception,
    and for linking STDP to NMF. Your work was a great inspiration to me.
    
    I must also thank my funding sources, without whom this research would not
    have been possible: the National Science Foundation (NSF),
    the Defense Advanced Research Projects Agency (DARPA),
    Qualcomm Technologies Inc., Intel Corporation, 
    Northrop Grumman Aerospace Systems,
    and the Swiss-American Society (SFUSA).
    
    Finally, I would like to thank my fianc\'{e}e, family, and friends 
    for all their love and support,
    and for keeping me sane throughout these years.
}


% Some custom commands for your list of publications and software.
\newcommand{\mypubentry}[3]{
  \begin{tabular*}{1\textwidth}{@{\extracolsep{\fill}}p{4.5in}r}
    \textbf{#1} & \textbf{#2} \\ 
    \multicolumn{2}{@{\extracolsep{\fill}}p{.95\textwidth}}{#3}\vspace{6pt} \\
  \end{tabular*}
}
\newcommand{\mysoftentry}[3]{
  \begin{tabular*}{1\textwidth}{@{\extracolsep{\fill}}lr}
    \textbf{#1} & \url{#2} \\
    \multicolumn{2}{@{\extracolsep{\fill}}p{.95\textwidth}}
    {\emph{#3}}\vspace{-6pt} \\
  \end{tabular*}
}

% Include, at minimum, a listing of your degrees and educational
% achievements with dates and the school where the degrees were
% earned. This should include the degree currently being
% attained. Other than that it's mostly up to you what to include here
% and how to format it, below is just an example.
\curriculumvitae
{

\textbf{EDUCATION}
  
  \begin{tabular*}{1\textwidth}{@{\extracolsep{\fill}}lr}
    \textbf{Doctor of Philosophy in Computer Science} & \textbf{2016} \\
    \vspace{6pt}
    University of California, Irvine & \emph{Irvine, CA} \\
    \textbf{Master of Science in Biomedical Engineering} & \textbf{2011} \\
    \vspace{6pt}
    ETH Zurich & \emph{Zurich, Switzerland} \\
    \textbf{Bachelor of Science in Electrical Engineering} & \textbf{2009} \\
    \vspace{6pt}
    ETH Zurich & \emph{Zurich, Switzerland} \\
  \end{tabular*}

\pagebreak

\vspace{12pt}
\textbf{RESEARCH EXPERIENCE}

  \begin{tabular*}{1\textwidth}{@{\extracolsep{\fill}}lr}
    \textbf{Graduate Student Researcher} & \textbf{2012--2016} \\
    \vspace{6pt}
    University of California, Irvine & \emph{Irvine, CA} \\
    \textbf{Summer Research Intern} & \textbf{2015} \\
    \vspace{6pt}
    IBM Research - Almaden & \emph{San Jose, CA} \\
    \textbf{Summer Research Intern} & \textbf{2013} \\
    \vspace{6pt}
    Fraunhofer IPA & \emph{Stuttgart, Germany} \\
    \textbf{Junior Specialist} & \textbf{2011--2012} \\
    \vspace{6pt}
    University of California, Irvine & \emph{Irvine, CA} \\
  \end{tabular*}

\pagebreak

\vspace{12pt}
\textbf{TEACHING EXPERIENCE}

  \begin{tabular*}{1\textwidth}{@{\extracolsep{\fill}}lr}
    \textbf{Teaching Assistant} & \textbf{2013--2015} \\
    \vspace{6pt}
    University of California, Irvine & \emph{Irvine, CA} \\
    \textbf{Tutorial Assistant} & \textbf{2009--2010} \\
    \vspace{6pt}
    ETH Zurich & \emph{Zurich, Switzerland} \\
  \end{tabular*}

\pagebreak

\textbf{BOOKS}

  \mypubentry{OpenCV with Python Blueprints: Design and develop advanced computer vision projects using OpenCV with Python.}{2015}{M. Beyeler. Packt Publishing Ltd., London, England, 230 pages, ISBN 978-178528269-0.}

\textbf{REFEREED JOURNAL PUBLICATIONS}

  \mypubentry{3D visual response properties of MSTd emerge from an efficient, sparse population code}{under review}{M. Beyeler, N. Dutt, and J. Krichmar.}
  \mypubentry{A GPU-accelerated cortical neural network model for visually guided robot navigation}{2015}{M. Beyeler, N. Oros, N. Dutt, and J. Krichmar. Neural Networks 72: 75--87.}
  \mypubentry{Efficient spiking neural network model of pattern motion selectivity in visual cortex}{2014}{M. Beyeler, M. Richert, N. Dutt, and J. Krichmar. Neuroinformatics: 1--20.}
  \mypubentry{Categorization and decision-making in a neurobiologically plausible spiking neural network using a STDP-like learning rule}{2013}{M. Beyeler, N. Dutt, and J. Krichmar. Neural Networks 48C: 109--124.}

% \vspace{12pt}
\textbf{REFEREED CONFERENCE PUBLICATIONS}

  \mypubentry{CARLsim 3: A user-friendly and highly optimized library for the creation of neurobiologically detailed spiking neural networks}{2015}{M. Beyeler*, T.-S. Chou*, K.~D. Carlson*, N. Dutt, and J. Krichmar (*co-first authors). IEEE International Joint Conference on Neural Networks (IJCNN).}
  \mypubentry{Vision-based robust road lane detection in urban environments}{2014}{M. Beyeler, F. Mirus, and A. Verl. IEEE International Conference on Robotics and Automation (ICRA).}
  \mypubentry{Exploring olfactory sensory networks: simulations and hardware emulation}{2010}{M. Beyeler*, F. Stefanini*, H. Proske, C.~G. Galizia, and E. Chicca (*co-first authors). Biomedical Circuits and Systems Conference (BioCAS).}

\pagebreak

\textbf{INVITED PUBLICATIONS}

  \mypubentry{GPGPU accelerated simulation and parameter tuning for neuromorphic applications}{2014}{K.~D. Carlson, M. Beyeler, N. Dutt, and J. Krichmar. 19th Asia and South Pacific Design Automation Conference (ASP-DAC).}

\vspace{3pt}
\textbf{SOFTWARE}

  \mysoftentry{CARLsim 3}{http://www.socsci.uci.edu/~jkrichma/CARLsim/index.html}{Efficient, easy-to-use, GPU-accelerated library for simulating large-scale SNN models with a high degree of biological detail.}
  
  \vspace{3pt}
  
  \mysoftentry{Motion Energy}{https://github.com/UCI-CARL/MotionEnergy}{CUDA implementation of the Motion Energy model (Simoncelli \& Heeger, 1998).}
  
%   \vspace{3pt}
  
  \mysoftentry{VisualStimulus}{https://github.com/UCI-CARL/VisualStimulusToolbox}{Matlab toolbox for generating, storing, and plotting 2D visual stimuli related to neuroscience such as sinusoidal gratings, plaids, random dot fields, and noise.}

}

% The abstract should not be over 350 words, although that's
% supposedly somewhat of a soft constraint.
\thesisabstract
{
Animals use vision to traverse novel cluttered environments with apparent ease.
Evidence suggests that the mammalian brain integrates visual motion cues across
a number of remote but interconnected brain regions that make up a visual motion
pathway. Although much is known about the neural circuitry that is concerned with
motion perception in the \acf{V1} and the \acf{MT}, little is known about 
how relevant perceptual variables might be represented in higher-order areas of 
the motion pathway, and how neural activity in these areas might relate to the
behavioral dynamics of locomotion. The main goal of this dissertation is to
investigate the computational principles that the mammalian brain might be using
to organize low-level motion signals into distributed representations of 
perceptual variables, and how neural activity in the motion pathway might
mediate behavior in reactive navigation tasks. 
I first investigated how the aperture problem, a fundamental conceptual challenge
encountered by all low-level motion systems, can be solved in a spiking neural
network model of \ac{V1} and \ac{MT} (consisting of $153,216$ neurons and $40$ 
million synapses), relying solely on dynamics and properties gleaned from known
electrophysiological and neuroanatomical evidence, and how this neural activity
might influence perceptual decision-making.
Second, when used with a physical robot performing a reactive navigation task 
in the real world, I found that the model produced behavioral trajectories that
closely matched human psychophysics data. 
Essential to the success of these studies were software implementations that 
could execute in real time, which are freely and openly available to the 
community. 
Third, using ideas from the efficient-coding and free-energy principles, I
demonstrated that a variety of response properties of neurons in \acf{MSTd}
area could be derived from \ac{MT}-like input features. This finding suggests 
that response properties such as 3D translation and rotation selectivity, 
complex motion perception, and heading selectivity might simply be a by-product 
of \ac{MSTd} neurons performing dimensionality reduction on their inputs. 
The hope is that these studies will not only further our understanding of how 
the brain works, but also lead to novel algorithms and brain-inspired robots
capable of outperforming current artificial systems.
}
