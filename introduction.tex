\chapter{Introduction}

\section{Motivation}
\label{sec:intro|motivation}
Animals are capable of traversing novel cluttered environments with apparent ease.
In particular, terrestrial mammals use vision to routinely scan the environment, 
avoid obstacles, and approach goals.
Such sensorimotor tasks require the simultaneous integration 
of perceptual variables within the visual scene 
in order to rapidly execute a series of finely tuned motor commands.
Evidence suggests that these functions might be mediated 
by a visual motion pathway in the mammalian brain
\citep{Britten2008,Orban2008};
that is, a densely interconnected network of dedicated brain regions,
where visual information flows from
the visual cortex (including the \acf{V1}, the \acf{MT}, and the \acf{MST};
which will be the focus of this thesis)
along the dorsal stream onto the \acf{PPC} on to the motor cortex.
Understanding how humans and other animals integrate visual motion information
to adeptly move around in the environment 
would require simultaneously recording
from all of these areas at a fine temporal and spatial scale---all while the
animal is performing a behavioral task.

Whereas it is possible today to record from either individual cells
(e.g., using extracellular single-unit recordings) or from entire brain areas
on the order of $10^7$ neurons (e.g., using \ac{fMRI} or \ac{EEG}),
it is still difficult to record at the microcircuit ($10^2-10^4$ neurons)
and mesocircuit ($10^4-10^7$ neurons) scales,
which are believed to be the scales at which phenomena such as perception, emotion,
and understanding begin to take shape \citep{Alivisatos2012}.
As a result of these limitations, empirical findings typically provide
only limited insight into how activity in microcircuits and mesocircuits
might relate to visual motion perception and active behavior under
natural conditions.
On one hand, although there have been studies on the behavioral dynamics of 
steering in humans \citep{FajenWarren2003,WilkieWann2003}, 
only recently did researchers begin to ask
whether individual behavioral performance is reflected in specific
cortical regions \citep{Billington2013,Field2007}.
On the other hand, despite a wealth of data regarding the neural
microcircuitry concerned with the perception of self-motion variables
such as the current direction of travel (``heading'') 
\citep{BrittenVanWezel1998,DuffyWurtz1997,Gu2006}
or the perceived position and speed of objects
\citep{EifukuWurtz1998,Tanaka1993},
little research has been devoted to investigating how this
circuitry may relate to active steering control.
%
% These constraints pose severe problems for current empirical protocols and
% recording techniques.
% First, there is currently no recording technique that can simultaneously record
% from multiple brain regions at both high spatial and temporal resolution.
% For example, high spatial resolution can be achieved with electrophysiological
% recordings (e.g., extracellular single-unit recordings, patch clamps), which can 
% measure the activity of one or several neurons.
% This technique is limited, however, to isolated cells in a single brain area.
% In order to measure activity on a brain-wide scale, the most commonly used
% experimental techniques include \ac{fMRI}, \ac{EEG}, and \ac{MEG}, which can achieve
% high temporal resolution, but have poor spatial resolution.
% Each electrode in an \ac{EEG} experiment, for example, is likely to pick up the 
% activity of $10^7$ neurons.
% Second, it is difficult to perform intricate recordings when an animal is behaving.
% Magnets used in \ac{fMRI} are large and immobile, and neurophysiological equipment
% runs the risk of being misplaced or damaged under free-movement conditions.
% Thus, most behavioral studies are restricted to psychophysical paradigms.
% % There are a few exceptions, however, like the experimental setups of the
% % Wurtz and Angelaki labs, which place subjects on a primate chair that is mounted
% % onto a motorized sled. This way, it is possible to record from brain regions of
% % interest while the animal is being physically displaced, which has led to some
% % fundamental insight into the cortical processing of optic flow during self-movement
% % \citep{Gu2006,Takahashi2007,Page2015}.
%

Meso-scale neural network simulations, on the other hand,
have the potential to overcome these limitations.
Thanks to recent developments in high-performance computing, 
it is now possible to simulate \acfp{SNN} on the order of $10^5-10^6$ neurons
on a single off-the-shelf \acf{GPU} in close to real time.
By taking into account computational and organizational principles of cortical
circuits, gleaned from electrophysiological and neuroanatomical evidence,
we can gain insight into the guiding computational principles that make these
networks so powerful.
In contrast to empirical experiments, computer simulations allow to
monitor and manipulate any variable in any neuron or synapse.
This is a powerful approach that could allow
neuroscientists to eventually build a mechanistic understanding
of meso-scale brain architectures and their functional importance,
and computer scientists could be able to extract the underlying computational
principles and condense them into abstract algorithms.
Second, by integrating these models into autonomous brain-inspired robots,
we can study (in real time) the link between neural circuitry 
and a system's behavior in specific tasks, 
offering new theories of brain function and
stimulating future experimental research.



\section{Contribution}
\label{sec:intro|contribution}

The main goal of this thesis is to investigate the computational principles that
the mammalian brain might be using to accurately and efficiently integrate
low-level motion signals into coherent percepts, 
how inferred perceptual variables might be represented at the population level,
and how neural activity in these populations might relate to behavior in
visually guided navigation tasks.

This research encompasses the design of computational models that reason about
visual motion information in a way that is comparable to the mammalian brain,
the development of open-source software libraries 
that provide the means to study these models,
and the conception of robotic systems that can evaluate 
both theories and simulations in a real-world environment.
As a result of these efforts, the contributions of this thesis span multiple
disciplines including computer science, computational neuroscience, and robotics.

In specific, the contributions of this thesis include:
\begin{itemize}
\item A large-scale \ac{SNN} model of cortical areas \ac{V1} and \ac{MT}, 
    whose simulated neural activity replicates data recorded \emph{in vivo},
    and whose behavioral response emulates human choice accuracy 
    in a motion discrimination task.
	The model demonstrates how the aperture problem, a fundamental conceptual
    challenge encountered by all low-level motion systems, can be
	solved in a biologically detailed \ac{SNN},
	relying solely on dynamics and properties 
    gleaned from known electrophysiological and neuroanatomical evidence,
    and how this neural activity might influence perceptual decision-making.
    
\item The deployment of said \ac{SNN} model on a robotics platform, used to steer
	the robot around obstacles while approaching a visually salient target,
    making steering decisions based solely on
    simulated neural activity generated from visual input.
    The study demonstrates how a model of \ac{MT} might build a 
    cortical representation of optic flow in the spiking domain, 
    and shows---for the first time---that
    these simulated motion signals are sufficient to steer a physical robot on
    human-like smooth paths around obstacles in the real world.

\item A computational model of optic flow processing in \acf{MSTd}, 
	demonstrating that a wide range of neuronal response properties of \ac{MSTd}
    neurons can be derived from \ac{MT}-like input features,
    based on a dimensionality reduction technique known as \acf{NMF}.
    This finding suggests that response properties such as 3D translation and
    rotation selectivity, complex motion perception, and heading selectivity
    might simply be a by-product of \ac{MSTd} neurons performing dimensionality
    reduction on their inputs.
    
\item CARLsim 3, an efficient open-source software environment developed as a group
	effort in the \ac{CARL} at the University of California, Irvine. 
    This C/C++ software is designed to aid in the conceptualization, utilization, 
    and research of brain circuit
    models of up to $10^6$ neurons and $10^8$ synapses by leveraging 
    off-the-shelf NVIDIA \acp{GPU}.
    
\item An efficient and scalable open-source software implementation of the
	Motion Energy \citep{SimoncelliHeeger1998}, the \emph{de facto}
    standard cortical model for visual motion integration.
    This software outperforms Simoncelli and Heeger's own implementation by
    orders of magnitude in terms of computational speed and memory usage,
    which enabled the practical feasibility of the empirical studies mentioned above.
\end{itemize}

The hope is that these studies will not only further our understanding of
how the brain works, but also lead to novel algorithms and computing systems
capable of outperforming current artificial systems.


\section{Organization}

The remainder of this thesis is organized as follows.

Chapter~\ref{ch:background} provides some theoretical background on computational
vision as well as the visual motion pathway in the mammalian brain.
First, general conceptual challenges in visual motion processing are briefly 
reviewed---such as the correspondence problem and aperture problem---and
possible mechanistic solutions for these problems are discussed.
Second, a brief overview is given  of cortical areas 
that are involved in visual motion processing in the mammalian brain.
Third, different approaches to modeling the brain are discussed.

Chapter~\ref{ch:SNN} introduces CARLsim, the open-source software framework
used in Chapters~\ref{ch:MT} and \ref{ch:ABR} to study the
cortical mechanisms of visual motion integration at the neural-circuit level.
The chapter gives an overview of CARLsim's user interface,
highlights architectural and algorithmic challenges overcome during development,
summarizes its computational performance, and discusses related work.

Chapter~\ref{ch:ME} introduces an efficient and scalable software implementation 
of the Motion Energy model by \cite{SimoncelliHeeger1998}.
The chapter gives both a conceptual explanation of the model as well as an overview
of the software's user interface.
The software's computational performance is also analyzed.
The development of this software, in combination with CARLsim, enabled the
practical feasibility of the studies described in Chapters~\ref{ch:MT} and
\ref{ch:ABR}.

Chapter~\ref{ch:MT} describes results from an \ac{SNN} model of visual motion
integration, modeled after cortical areas \ac{V1} and \ac{MT}.
The purpose of this model was to understand how the aperture problem could be
solved in a biologically detailed \ac{SNN},
relying solely on dynamics and properties gleaned from known electrophysiological
and neuroanatomical evidence.
The chapter summarizes results from both simulated neural activity as well as
behavioral performance in a motion discrimination task.

Chapter~\ref{ch:ABR} describes the deployment of the \ac{SNN} model outlined in
Chapter~\ref{ch:MT} on a robotics platform, and summarizes behavioral results
of the robot performing a visually guided navigation task.
The purpose of this model was to investigate how a simulated cortical representation
of optic flow in \ac{MT} might relate to perceptual decision-making 
as well as active steering control.
The chapter summarizes results from both simulated neural activity as well as
behavioral performance in a reactive navigation task.

Chapter~\ref{ch:MSTd} introduces a novel computational model of optic flow
processing in \ac{MSTd}, based on principles of the \acf{ECH}.
The purpose of this model was to demonstrate that a wide range of neuronal
response properties of \ac{MSTd} neurons could be derived from \ac{MT}-like
input features, suggesting that these empirically observed properties 
might emerge from a statistically efficient representation of optic flow.
The chapter compares simulated neural activity to a variety of empirical findings,
and aims to provide some structure to the multiple and often conflicting
results from \ac{MSTd} electrophysiology and modeling studies.

Chapter~\ref{ch:conclusion} presents the main conclusions defended in this
thesis, discusses limitations of the work, and outlines possible avenues for
future studies.