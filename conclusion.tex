\chapter{Summary and Conclusion}
\label{ch:conclusion}

Simulating large-scale models of biological motion
perception is challenging, due to the required memory to store
the network structure and the computational power needed to
quickly solve the neuronal dynamics.
In Chapters~\ref{ch:SNN} and \ref{ch:ME} I introduced software that
enabled the practical feasibility of most of the research presented in
this thesis.
CARLsim allowed us to simulate meso-scale neural networks 
(on the order of $10^5$ neurons and $10^7$ synapses) that require 
a high degree of biological detail without sacrificing performance.
Analogously, development of an efficient and scalable implementation of
the Motion Energy model \citep{SimoncelliHeeger1998} was necessary in order
to handle video streams ($\sim 360 \times 480$ pixels) of up to 
$30$ frames per second in (quasi) real time.
These software efforts allowed us to embody the computational models
presented in this thesis on a neurorobotics platform so that we could
investigate the link between simulated neural circuitry and real-world
robot behavior.
All software was released under open-source licenses,
making it freely and openly available for other researchers to use the
software, adapt it for their purposes, and expand for future studies.

In Chapter~\ref{ch:MT}, I presented a large-scale spiking model 
of visual area \ac{MT} that 1) was capable of exhibiting both component and
pattern motion selectivity, 2) generated speed tuning curves that were in
agreement with electrophysiological data, 
and 3) emulated human choice accuracy and the effect of motion strength 
on reaction time in a motion discrimination task.
Although the model was able to demonstrate how the aperture problem could
be solved with solely biologically plausible mechanisms and network dynamics,
future studies could investigate how the model might generalize to more
perceptually challenging stimuli that include transparent motion,
second-order motion, or binocular cues.

In Chapter~\ref{ch:ABR}, I described the deployment of the model outlined
in Chapter~\ref{ch:MT} on a robotics platform, used to steer the robot
around obstacles toward a visually salient target.
The study demonstrated how a model of \ac{MT} might build a cortical
representation of optic flow in the spiking domain, and showed that these
simulated motion signals were sufficient to steer the robot on human-like
smooth paths around obstacles in the real world.
In the future, it would be interesting to see the model being extended
to cortical areas beyond \ac{MT}, in order to investigate the functional
real-world implications of visual motion processing in higher-order areas
of the motion pathway.

Finally, Chapter~\ref{ch:MSTd} introduced a novel computational model
of optic flow processing in \ac{MSTd} based on ideas from the 
efficient-coding and free-energy principles. 
I demonstrated that a variety of visual response properties of \ac{MSTd}
neurons could be derived from \ac{MT}-like input features using \ac{NMF},
suggesting that response properties such as 3D translation and rotation
selectivity, complex motion perception, and heading selectivity might 
simply be a by-product of \ac{MSTd} performing dimensionality reduction 
on their inputs.
Despite its simplicity, the model was able to explain a variety of
\ac{MSTd} visual response properties, ranging from single-unit activity to
population statistics.
These findings provide a further step towards a scientific understanding
of the often nonintuitive response properties of \ac{MSTd} neurons.
In the future, this work might lead to new intellectual property (patent pending)
and find embodiment in brain-inspired robots and drones.
The model itself might be extended to account for nonvisual response
properties of \ac{MSTd} neurons, which include the involvement of
vestibular and eye movement-related signals.

Overall, the contributions of this thesis span multiple disciplines including
computational neuroscience, computer science/engineering, and robotics.
The present work might be of interest to the neuroscience community, as it
1) provides a mechanistic description of visual motion processing in the brain
and 2) makes specific predictions that can be empirically verified.
The present work might also be of interest to the computer science and
engineering communities, as it 
1) sheds light on the computational principles that might be at play 
in the visual brain, 
2) makes significant contributions to the open-source movement, and
3) describes algorithms that are compatible with recent neuromorphic hardware.
Finally, the present work might be of interest to the robotics community, as it
1) investigates the functional link between models of brain function and an
agent's real-world behavior,
and 2) constitutes a first step into building fully functional and autonomous
neurorobotics platforms.

It is my hope that these studies will not only further our understanding 
of how the brain works, but also lead to novel algorithms and 
brain-inspired robots capable of outperforming current artificial systems.


